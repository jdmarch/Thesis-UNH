% Chapter Name
\chapter{\sc Introduction}
\label{ch:Introduction}
\pagenumbering{arabic} %Include ONLY for the first chapter

% --- Insert Text --- %
\section{Motivation}
The technology world has, over the past few decades, been dominated by the trend of constantly making devices smaller and smaller.  This happens at the macroscopic level whereby computers are put into smaller and smaller devices such as laptops, tablets, phones and watches, however, it also occurs within the devices themselves. The electronics, which make up the devices, continue to shrink in size well into the realm of the nanoscale where the characteristic length is on the order of less than 100nm. This trend has caused a revolution in science opening up new fields of study and unearthing a rich field of new physics, which only occurs at the nanoscale. While the downsizing of electronic components within a device may eventually reach inherent limitations in length scale, the trend of putting computers into more and more devices shows no indication of stopping. Without a doubt we live in a world that is increasingly reliant on electronics for day-to-day functionality in numerous aspects of life from healthcare to economics.

The push for fundamental research into physics at the nanoscale happened explosively due to large demand already present from consumers for faster and faster electronics as well as the remarkable diversity of fields in which fundamental nanoscience has effects. As new physics emerges in the new realm of the nanoscale, a natural reaction is to ask for a description of the environment with the highest possible resolution: nano or sub-nanoscale microscopy. It is worth noting here that the original push towards atomic resolution imaging techniques was a coevolution of new imaging technology combined with advances in vacuum technology that provided a clean room for nanoscale experimentation. Many of the techniques that will be described later in this thesis would not be possible were it not for great advancements in the 1960's and 1970's in Ultra High Vacuum technology.

Exploring and probing the physics at the nanoscale and below sheds light on a wide variety of material properties from electronics, optics and magnetism, to fundamental chemical and biological properties like catalysis. Furthermore it happens that manipulation of materials at the nanoscale, namely individual atoms, molecules, or small clusters, can give rise to new types of materials altogether which were previously unknown or thought to be unstable - \emph{e.g.} graphene and Bose-Einstein Condensates

As nanoscience continues to expand, scientists will not only push the limits of observable length scales but will also learn to harness the unique phenomena which occur only at this scale, quantum effects and other unique electrical, mechanical, chemical and biological properties. Nanoscience is indeed a field that encompasses nearly every discipline of science and engineering. [intro to nanoscale science and technology - di Ventra]. With resolution for imaging already available in the sub-angstrom regime, in the future one avenue that will certainly be pursued is improving resolution in the time domain. Novel developments in electron optics will allow for surface imaging with femtosecond resolution. High resolution time-domain imaging coupled with high spatial resolution will allow surface dynamics to be studied in near real-time. This will allow important physical, chemical, and biological processes such as catalysis in chemical reactions or enzyme functionality in biologic systems to be probed at a local scale with an accuracy in the time domain never before seen.

\section{Organic Electronics}
A second shift in the electronics industry, which has happened more recently, is the focus on research and applications of organic electronics. The premise of organic electronics is to create devices out of carbon based materials. A number of benefits arise from using carbon as the basis of electronic devices when compared to traditional silicon based electronic technology.

Conventional electronics are, for the most part, based around metallic conductors and silicon based semiconductors. While the traditional semiconductor manufacturing industry is well formed and highly developed, silicon based technology has some inherent limitations. First and foremost, silicon and other metals must be mined from the earth's crust whereas many carbon-based materials can be directly synthesized in a standard organic chemistry lab.  Second, most standard conductors and semiconductors used in common electronic devices are rigid and inflexible. Carbon-based electronics opens up the possibility of flexible electronics such as flexible computer displays or mobile phone screens. Carbon based electronics may also prove to be more efficient in dealing with the thermal load of operation at the nanoscale which is becoming more and more of an obstacle for cutting edge transistor scales [source: heat dissipation in carbon nanotube transistors AIP]

The participants of the 2012 Chemical Sciences and Society Summit (CS3) laid out their vision for the future of organic electronics and their impact on society that cover the aforementioned limitations of traditional semiconductor manufacturing briefly summarized here \cite{CS3}:
\begin{enumerate}
\item
Organic electronic devices will do things that silicon-based electronics cannot do.

\item
Organic electronic devices will be more energy-efficient and eco-friendly.

\item
Organic electronic devices will be manufactured in a more resource-friendly and sustainable fashion
\end{enumerate}

While there is currently much research underway concerning novel organic electronic materials and their properties, one large obstacle, which still remains, is industrial scale manufacturing. However this is being overcome somewhat as more and more electronic devices come to market utilizing organic electronic technologies manufactured by industry leaders such as Samsung, LG, etc. A flagship device that has reached a large portion of the market is the Samsung Galaxy series of mobile phones and devices. Many iterations of the Galaxy line of utilize organic based semiconductors to create light emitting diodes that make up the device's display. 

OLEDs work in a very similar manner to normal LEDs with the main difference being that the active region for light emission is created from multiple layers of carbon-based material. [OLED DIAGRAM] In a similar manner to LEDs, organic materials are being studied for usage in photovoltaic devices, OPVs. Light enters an OPV through the transparent electrode and interacts with the active region of carbon-based materials. If the photon has sufficient energy, an excited state in the active region is created which manifests as an electron hole pair called an exciton. Due to the mismatch in band structure between the acceptor and donor materials an effective electric field is created. Interaction with the field causes the exciton to split into free particles. Electrons in the donor material conduction band exist at a higher energy level than the acceptor material conduction band and thus move to the acceptor material. Overlap between the acceptor material conduction band and the electrode conduction band allows the electron to be extracted to perform work in an external load before neutralizing with the holes through the alternate electrode.

In both of these types of devices, OLEDs and OPVs, organic materials are being considered both for the active device regions (materials which will emit or adsorb light) as well as for electrode materials. One type of material needed in both devices is a transparent conductor. The transparent conductor functions as the topmost electrode in the device, which can conduct electricity while simultaneously letting light penetrate into or emit from the device. Current materials used for transparent electrodes are often costly to produce and limit the price of a device.

\section{Graphene}
Graphene is an allotrope of carbon consisting of a single layer of sp$^2$ hybridized carbon atoms with a hexagonal crystal structure. First isolated in 2004, graphene became the first truly 2D material and hence spawned an explosion of research into the possibility of other 2d materials. Previously thought to be unstable, graphene was found to have a plethora of interesting properties ranging from electrical to mechanical and seemingly everything in-between. First isolated in 2004, graphene became the first truly 2D material and hence spawned an explosion of research into the possibility of other 2D materials. Previously thought to be unstable, graphene was found to have a plethora of interesting properties ranging from electrical to mechanical and seemingly everything in-between.

The structure of graphene arises from the hybridization of the carbon electron orbitals. The sp$^2$ hybridization happens when a carbon atom is bonded to three other atoms; in the case of graphene the three other atoms are also all carbon. The bond hybridization results form the mixing of a carbon s-orbital with two carbon p-orbitals to create a set of three sp$^2$ hybrid orbitals. The bond geometry of sp$^2$ hybrid orbitals is planar with a 120$^{\circ}$ angle between bonds. This bond structure then suggests graphene adopts a hexagonal crystalline lattice where the C-C bond length is 1.42{\AA} while the hexagonal lattice constant is larger by a factor of square root of three at 2.46{\AA}. [GRAPHENE STRUCTURE DIAGRAM]

Quite soon after its isolation, a number of interesting properties of graphene were elucidated. Graphene was found to have very high electron mobility, in fact of the highest mobility values ever recorded [cite WOLF]. A single graphene sheet is also nearly perfectly transparent with less then 3{\%} of white light being adsorbed. This combination of high electron mobility coupled with high transparency makes graphene an ideal candidate in organic electronic devices that rely on either light emission or adsorption. The top layer of an OLED or an organic photovoltaic (OPV) device could be crafted out of graphene so that it would function as a transparent conductor. Current materials used for transparent conductors are generally expensive to manufacture and or require more exotic materials to function such as Indium-Tin-Oxide (ITO) - using graphene for transparent electrodes would be one straightforward way to reduce costs as well as weight in organic electronic devices.

In terms of structural properties, graphene is of interest for use as a template for self-assembled growth of organic materials. Graphene by itself is nearly perfectly planar yet when grown atop certain substrates it adopts a periodically corrugated structure. The corrugation arises due to the lattice mismatch between graphene and its growth substrate if the substrate also adopts a hexagonal lattice. This phenomenon is known as a moire pattern and is caused by two geometrically similar lattices overlaid atop one another with a slight lattice mismatch between the two. Moire patterns can also be caused by rotational mismatch between two geometrically similar lattices. 

The corrugation inherent in the carbon layer for graphene grown on hexagonal crystals like ruthenium can then act as a template for the growth of other materials such as organic semiconductors like pentacene derivatives. My thesis work focuses on a way to modulate the periodic corrugation that arises when graphene is grown epitaxially on the hexagonal ruthenium 0001 surface. When atomic hydrogen is introduced into the graphene growth process additional low energy diffraction peaks can be seen that would normally not be present. Scanning tunneling microscopy results confirm the presence of many different moire structures that are normally not observed when graphene is grown atop ruthenium without the presence of atomic hydrogen. These moire structures have a wide variety of characteristic lengths, the moire periodicity, ranging from 1nm to 3nm, with 3nm being the standard periodicity that graphene adopts on Ru(0001) \cite{march}.

\section{Self-Assembly}
In nanoscience there are generally two main categories of fabrication techniques when it comes to device production. One method involves starting from the top with a bulk material such as a silicon wafer then selectively cutting away or carving out areas of unnecessary material by lithographic etching into a specific shape. This is known as the `top-down' approach. The second way is to begin with only a substrate material and create a device by assembling individual components layer by layer or even atom by atom, this is known as the `bottom-up' approach.

Self-assembly is a bottom-up approach to nanoscale design whereby individual building blocks (in this case atoms, molecules, or clusters of atoms) arrange themselves into a natural pattern without aid of external forces. There are numerous examples of self-assembly in nature: proteins that fold into specific functional patterns, organic materials which form complex layered networks to guide growth of crystals in mollusks and crustaceans, assembly of virus protein sub-units into a full virus by random motion, etc.

In the lab, self-assembly can be harnessed to create layers of patterned molecules for use in devices. Controlling pattern formation by molecular functionalization allows the scientist to create nearly any type of structure without the hassle of having to manipulate each individual atom or building block. A better understanding of what causes the self-assembly process to happen at the nanoscale level can advance this bottom-up method. Ultimately the study of the underlying physics at the surface is necessary to characterize the processes involved in self-assembly.

\section{Tools}
The UNH surface science lab is uniquely equipped to probe fundamental physics at the nanoscale level and specifically to advance understanding of the physics that governs molecular interactions at interfaces between surfaces.

The primary tool this thesis work utilizes is scanning tunneling microscopy (STM). STM is a non-optical scanning probe microscope that generates images of the local density of electronic states in a surface based on the concept of quantum tunneling. Quantum tunneling has no classical analogue and is a phenomenon restricted solely to the nanoscale regime.

STM is a powerful tool for probing the real space structure of surfaces with incredibly high resolution in all directions. With STM the exact locations of individual atoms can be inferred from their imprint in the map of LDOS and thus in essence STM images can be considered a topographic map of the surface with sub angstrom resolution. The ultra-high resolution available from STM is crucial in studying pattern formation in self-assembled systems or other highly ordered crystalline surfaces. The microscope I have conducted much of my thesis experiments on is a homebuilt variable temperature scanning tunneling microscope, VT-STM, which is integrated in an ultra high vacuum, UHV, system capable of many types of surface analysis and sample preparation.

The second primary tool utilized in this thesis work is Low Energy Electron diffraction, LEED. The LEED system in our UHV chamber (Omicron SpectaLEED) can be used for reciprocal space imaging of crystalline surfaces as well as operating in Auger Electron Spectroscopy mode to analyze surface contaminants.

The rear-view LEED system illuminates the sample surface with a collimated beam of monochromatic electrons. The incident beam energy is variable from 0-2 keV, however, most of my LEED work is focused on the lower energy regime 40-100eV. Electrons in this low energy regime have a very low mean free path within the bulk crystal sample, thus they are highly sensitive to the surface layers of the crystal.

The patterns displayed on the LEED screen can be considered a scaled representation of the reciprocal space surface lattice. This type of imaging technique is useful for understanding qualitatively the ordering of the sample surface. A well ordered crystal substrate will have a specific diffraction pattern related to the geometry of the surface. A disordered or contaminated crystal surface will have a less well defined diffraction pattern and may even exhibit extra diffraction spots based on adsorbate contaminants. LEED will primarily be used in this thesis work as a guidance tool for characterizing the state of the sample surface before beginning an STM experiment.

One of the restrictions of conventional LEED systems is that the electron source blocks the (0,0) spectrally reflected beam from visibility in the diffraction images. At very low energies only the spectral beam would be seen as the other diffracted beams occur at diffraction angles that place them off the screen. Thus conventional LEED systems have a lower limit in their minimum energy accessible around 40 eV. 

Of interest to my thesis work is the behavior of reflected electrons from the graphene surface in very low energies, $E < 20$ eV. There has been some previous work to suggest that oscillations in low energy electron reflectivity curves have a relation to the thickness of the film they reflect from and thus could be used as a gauge for layer thickness. To bypass the restriction of conventional LEED systems I utilize a third analysis tool for my research, low energy electron microscopy, (LEEM).

LEEM instruments are a subset of cathode lens microscopes where the sample being probed acts as part of the cathode lens itself. In LEEM the electron beam optics are not required to be directly in line with the sample or the detection optics which allows for alternate geometries through the sue of magnetic beam splitters. Alternate beam line geometries allow the spectral beam to be imaged at all energies so the 0-20 eV range can be covered with little issue. An added benefit of LEEM is the ability to rapidly switch which optical plane is being projected into the imaging column of the microscope thus allowing the LEEM to switch from real-space imaging to reciprocal space imaging with little downtime in between. While LEEM can't provide the lateral resolution as low as an STM, the ability to probe small domains in both real and reciprocal space is a very powerful tool. 

\section{Summary}
This thesis outlines a novel approach to the study of nanoscience by analyzing and characterizing a new growth method for varying materials that will inevitably be part of future organic based electronic devices, namely graphene and planar organic semiconducting molecules like pentacene derivatives.


We have developed a growth method for graphene layers atop a ruthenium crystal substrate that yields a periodically corrugated graphene domains with a variable moire spacing previously unreported. The underlying cause of these shorter period domains will be analyzed through a variety of imaging techniques with an emphasis on the structural properties of the graphene layers.

Finally I will show how control over the structure of the graphene substrate could affect the way organic molecules naturally assemble themselves into distinct patterns. The modulation of the graphene templates could then in the future be a method for controlling the directed self-assembly of organic molecules and metallic nanoclusters. Self-assembled monolayers of organic and metallic materials will be useful in the future in organic electronic devices such as photovoltaic devices or quantum dot devices. 

Understanding the basic physics that governs the flow of electricity though organic devices is important for characterizing device efficiency. One possible way to improve efficiency in organic electronic devices could be through enhanced charge transport across the layers within the device as well as enhanced charge transport within the layers of a device. Both of these could be addressed by improving the ordering of the layers at the molecular level through directed self-assembly for bottom-up growth methods. 

\begin{itemize}
\item Chapter 1, \emph{Introduction}, 
\item Chapter 2, \emph{Graphene},
\item Chapter 3, \emph{Tools for Surface Science}
\end{itemize}

% ------------------- %