% Chapter Name
\chapter{\sc Conclusions}
\label{ch:Conclusions}

In this thesis I have described the growth process for a novel graphene system, which exhibits unique structural properties when compared to conventional graphene systems. The study of the graphene surface structure requires a wide variety of tools and techniques in order to provide an atomic scale description of the various reconstructions present. To add additional data complementary to what can be collected at the UNH Surface science lab, a novel microscopy technique, LEEM, has been employed to study graphene as well as other two-dimensional materials in collaboration with Sandia National Laboratory's Center for Integrated Nanotechnology and Brookhaven National Laboratory's Center for Functional Nanomaterials. To further our ability to process data from these experiments, a software package has been designed to aid in the extraction of relevant data from LEEM and LEED experiments. The software is designed to be cross-platform and open source in order to reach as large of an audience as possible.

The materials of interest in this work are primarily restricted to organic materials and/or 2D materials. These types of material offer great promise for future design of electronic devices in a wide gamut of applications. Specifically, adoption of organic materials into everyday electronic architectures may allow for higher electronic efficiency, lower cost, and a more environmentally friendly production. Furthermore, organic and 2D materials provide an avenue for flexible electronics, which then extends current device architectures to novel form factors such as flexible solar panels and wearable electronics.

Many of the novel properties expressed by the materials studied in this work stem directly from the restricted nature of the dimensionality. Analysis of materials and devices at the nanoscale and beyond requires many different tools to characterize the structure, electronic effects, and novel physics which emerge at atomic length scales. Studies of the fundamental properties of new materials will provide guidance to future applications. Furthermore, advances in growth techniques for the ever growing set of stable 2D materials, will provide opportunities for unique combinations of new materials to create heterostructures with tailored properties.

\section{Summary of Results}
The growth of a novel hydrogen-graphene system displaying unique moire domain polymorphism has been detailed in the previous chapters. Numerous studies have been carried out to better understand the structure of this system. Auger electron spectroscopy reveals no contaminants to the system above roughly 1\% of surface atoms, from which I can conclude that the introduction of atomic hydrogen to the growth process results in the structural changes observed. Characterization by conventional LEED reveals a set of extra diffraction peaks not present in the standard graphene/ruthenium system, which form a $(2\sqrt{3} x 2\sqrt{3})R30^\circ$ diffraction pattern relative to the primary ruthenium (1x1) pattern. Analysis with STM confirms that many areas on the sample exhibit moire periods which are not present in conventionally grown graphene on ruthenium. Rather than one domain consisting of a 2.98 nm moire period, many different domains have been observed with moire periods ranging from 1nm to 2.98nm. These domains are found to coexist on the same sample with surface areas ranging from a few square nanometers to hundreds of square nanometers.

Many of the individual moire domains have been imaged with atomic resolution. From this analysis, I find a linear relationship between the moire size observed and the number of ruthenium unit cells present along the superstructure edge. This suggests that the transition from one moire domain to the next largest domain corresponds to the addition of one ruthenium unit cell to the superstructure size. Previously the graphene-ruthenium superstructure size has been modeled as a (n+1) graphene unit cells atop n ruthenium unit cells. However, some domain sizes observed via STM do not fit this model. Instead, I suggest that the larger model provided by surface X-ray diffraction experiments, provides a better fit to the entire collection of observed domain sizes. In this model the graphene-ruthenium superstructure is larger by a factor of two and can be described as (n + 2) graphene unit cells atop n ruthenium unit cells. This superstructure is made up from four separate inequivalent superstructures of the size governed by the other model. This results in a 25 x 23 superstructure size to reproduce the observed moire pattern of 2.98 nm seen in the standard graphene on ruthenium system. Specifically, the larger superstructure model fits observed moire domains such as 2.56 nm, which did not fit the previous model under the assumption that n must be an integer.

A preliminary study of the polymorphic graphene system using LEEM at the Sandia National Laboratory Center for Integrated Nanotechnology was conducted to gather information concerning the carbon layer thickness resulting from this growth process. However, due to complications with the microscope possibly arising from poor sample quality, sufficient data was not available quantify the number of layers in the system.

A second study of the graphene-ruthenium system and its interaction with hydrogen was completed at the Brookhaven National Laboratory Center for Functional Nanomaterials. This study focused on characterization of the growth process of graphene on ruthenium and the influence of hydrogen. At room temperature, little to no difference was observed when growing graphene in the presence of atomic hydrogen. Given the nature of the interaction between hydrogen and ruthenium, it was difficult to quantify the amount of atomic hydrogen arriving at the ruthenium surface. Its possible that the partial pressure of atomic hydrogen in proximity to the ruthenium surface was too low to make a meaningful contribution to the growth process. Restrictions resulting from the geometry of the main LEEM chamber prevented shortening the distance between the hydrogen cracking device and the sample. While the role of hydrogen in the polymorphic graphene system remains an unanswered question, the second experiment was able to provide direct evidence that the electron reflectivity curves can be used to map the carbon layer thickness in the sample. LEEM-I(V) data was collected from clean ruthenium, single layer graphene, and bilayer graphene. Distinct differences in the shape of the I(V) curves in the very low energy regime were found. This data also suggests that the first LEEM experiment likely has large portions of the sample surface exhibiting bilayer graphene characteristics.

One of the goals in this work was to characterize the polymorphic graphene system and better understand the role of hydrogen in the graphene growth dynamics. Since we were unable to replicate the growth process used at UNH directly in the LEEM systems used for study, the influence of hydrogen still remains an open question. However, a direct result from these experiments was the development of a software package, PLEASE, to aid in analysis of LEEM and LEED experimental data. More details on the software are given in Appendix A. This software package was then used to further the analysis of experimental data from many other types of 2D materials such as $MoS_2$, Black Phosphorus, $SnSe_2$ and $MoTe_2-W alloy$. Many of these materials display interesting electronic characteristics, thus an understanding of their surface structures at the atomic level is useful. Using experimental data collected at the BNL CFN, the custom I(V) extraction software described in Appendix A, and dynamic electron multiple scattering modeling, the surface structure of 2H-$MoS_2$ has been determined with sub-angstrom resolution demonstrating that the surface structure is distinct from the known bulk lattice structure \cite{mos2-surfsci}. Further work detailing the surface structure of Black Phosphorus is currently awaiting publication.

\section{Future Outlook}
There are a number of remaining questions concerning the novel graphene system presented in this work. Further studies of the system with low-temperature STM may provide additional detail to the atomic picture of the surface structure. Specifically, studying the domain boundaries between area with differing moire periods with atomic resolution may provide insight as to how one domain transitions to another. Furthermore LT-STM may make electronic characterization of the system easier. Aside from unique structural differences between polymorphic graphene and standard graphene on ruthenium, there may also be unique electronic properties. If the hydrogen in the system passivates the ruthenium surface, then the electronic properties of the polymorphic system should be closer to that of quasi-freestanding graphene (QFG) similar to the use of hydrogen to generate QFG on silicon carbide \cite{SiC-passivation}.

Utilization of the graphene periodic moire structure as a scaffold for growth of other materials also remains an interesting area for future research. Previously, deposition of metal atoms atop the moire surface has demonstrated that metal atoms tend to grow in clusters centers on the moire minima as would be expected envisioning the growth process similar to marbles in a muffin tin \cite{ptclusters}. This form of growth leads to a periodic array off metal clusters with the spacing dictated by the moire period. Polymorphic graphene then offers an avenue for generating arrays of nanoclusters with different  periods.

While LEEM provided a novel method for monitoring the graphene growth process in real-time, it proved trouble some to integrate with the polymorphic graphene growth process. The main concern is providing a sufficient atmosphere of atomic hydrogen to the ruthenium surface during the growth. A research proposal has been submitted to continue this project at the BNL CFN. Rather than monitoring the growth in real time, instead the growth process can be moved to the LEEM prep-chamber rather than the main imaging chamber. This allows a better replication of the growth conditions used at UNH by integrating a restively heated solid carbon source as the hydrogen cracking device in closer proximity to the sample surface. Results from this growth method can be then compared to the previous data collected so look for a signal of hydrogen influence. The smoking gun evidence being searched for is a change in the observed LEED pattern after the growth process indicating an influence from hydrogen assuming all other contaminants are kept to a minimum as monitored by AES.

Future experiments at the BNL CFN may also be useful for further electronic characterization of the system. As previously mentioned, the polymorphic graphene system may display electronic properties distinct from that of standard graphene on ruthenium. While LT-STM may provide insight to the electronic properties via STS, the AC-LEEM system at BNL is currently being integrated with the new synchrotron source, the National Synchrotron Light Source II. This system then allows both characterization with normal LEEM, PEEM, and LEED as well as with ARPES. Using ARPES, the band structure of the target material at the Fermi surface can be imaged directly from a micron sized area.

Finally, there is also much work left to be done to refine the PLEASE software package. Each additional data set made available for study leads to new features and fixes for many bugs. While the software is currently in a stable state designated as the version 1.0.0 release, there are a number of features still being worked on for future releases. More details on future plans for the software can be found in Appendix A.
